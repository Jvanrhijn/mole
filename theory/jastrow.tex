\documentclass{article}

\title{Jastrow Factor}

\author{Jesse van Rhijn}

\date{2019-02-20}

\usepackage{amsmath}

\begin{document}

\maketitle

The Jastrow factor employed in this library has the functional form

\begin{align}
  J(\mathbf{R}, \mathbf{r}) = \exp(f_{ee} + f_{en} + f_{een}),
\end{align}

with 

\begin{align}
  f_{ee}(r_{ij}) = \sum_{i=1}^{N_e}\sum_{j > i}^{N_e} 
    \left( \frac{b_1 r_{ij}}{1 + b_2 r_{ij}} + \sum_{p=2}^{N_b - 1} b_{p+1}r_ij^p \right).
\end{align}

The gradient and laplacian of this term are given by

\begin{align}
  \nabla_k f_{ee} = \sum_{i=1}^{N_e} \hat{r}_{ik}\left( \frac{b_1}{1 + b_2 r_{ik})^3}
    \sum_{p=2}^{N_b - 1} b_{p+1}pr_{ik}^{p-1} \right),
\end{align}

and

\begin{align}
  \nabla^2_k f_{ee} = \sum_{i=1}^{N_e} \frac{1}{r_{ik}}
    \left( \frac{-2 b_1 b_2}{(1 + b_2 r_{ik})^3}
    + \sum_{p=2}^{N_b - 1} b_{p+1}p(p-1)r_{ik}^{p-2}\right) \\ 
    \times (x_i - x_k + y_i - y_k + z_i - z_k).
\end{align}

We then have $\nabla f_{ee} = \sum_{k=1}^{N_e} \nabla_k f_{ee}$ and $\nabla^2 f_{ee} = \sum_{k=1}^{N_e}\nabla^2_k f_{ee}$. 


\end{document}
